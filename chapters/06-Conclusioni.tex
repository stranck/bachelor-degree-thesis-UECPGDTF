\documentclass[main.tex]{subfiles}

\begin{document}
\sloppy


\vspace{1.0cm}

\section{Conclusioni}\label{sec:End}
Il progetto è partito come una semplice espanzione con nuove funzionalità, ma è presto diventato necessario un refactoring generale alle parti più importanti. Questo processo di refactoring ha, purtroppo, portato via la maggior parte del tempo a disposizione per il progetto, riducendo il numero di nuove funzionalità implementate a due (Framing System ed Iris). In particolare, sono emersi problemi sulla parte di rendering dei vari moduli di una fixture, risolti creando una pipeline di rendering dinamica, e sulla gestione degli stessi, risolti riscrivendo la gerarchia di classi dei componenti in maniera che ciascuno debba definire solamente i suoi comportamenti specifici nei confronti del mondo reale. Sono state anche evidenziate criticità nella classe di interpolazione, ovvero la classe che si occupa di emulare il funzionamento di un vero motore, sistemate sostituendo l'algoritmo con uno più vicino a ciò che è implementato su una fixture reale.

Durante lo svolgersi del tirocinio ho potuto mettere in pratica ed analizzare il funzionamento nel mondo reale di pattern, conoscenze matematiche e conoscenze di computer graphics. Ho anche acquisito un discreto know-how sul funzionamento degli internals di Unreal Engine, e di come poter sviluppare (da un punto di vista tecnico) contenuti sullo stesso. 

\subsection{Problemi riscontrati}\label{subsec:6_Problems}
Durante lo sviluppo del plugin sono emerse numerose problematiche legate ad Unreal Engine. Prima fra tutte la scarsa documentazione e malagestione degli oggetti.\newline

Prendiamo in esempio la generazione della MaterialFunction utilizzata per la pipeline di rendering, in particolar modo la generazione del nodo di output. Nell'implementazione originale, il nostro codice creava un nuovo nodo (chiamando il costruttore) e poi lo usava direttamente. Questo portava ad uno strano bug: tutto funzionava fino a che UnrealEngine non veniva chiuso e riaperto. All'apertura, tutti i nodi collegati alla MaterialFunction della pipeline di rendering (Quindi dentro i materiali Light e Lens) venivano scollegati. Dopo numerosi giorni ad effettuare debugging sugli internals di Unreal Engine ed a effettuare reverse engineering sul processo di serializzazione e deserializzazione dei materiali, si è scoperto che il nodo di output della MaterialFunction aveva un id invalido, che non consentiva il ricollegamento in fase di caricamento del materiale. Questo perché il costruttore di una \lstinline{UMaterialExpressionFunctionOutput} non chiama automaticamente la funzione (non documentata) per la generazione dell'id di output. Questa mancanza può essere considerata legittima, a patto però che ci sia di supporto una documentazione solida in cui venga spiegato in maniera chiara il procedimento per la creazione via codice C++ di una MaterialFunction. Documentazione che, purtroppo, non esiste.

Questa problematica non è circoscritta solamente a quel caso, ma è stata una costante per praticamente tutto il progetto. In generale, tutorial, convenzioni, buone prassi e documentazione per scrivere un plugin per Unreal Engine sono materiali molto rari se non introvabili. Al contrario, è molto semplice trovare guide sulla programmazione di attori attraverso BluePrint. Sembra come se Unreal Engine, nonostante ne metta a disposizione la possibilità, non volesse o per lo meno non sia fatto per essere utilizzato scrivendo codice nativo in C++.
\newline

Un'altra problematica grossa è legata allo scarso ascolto delle richieste degli utenti (o almeno quelle specifiche). Durante lo sviluppo sono incorso in molti thread datati anche quasi 10 anni in cui utenti si lamentavano di feature utili mancanti, che però tutt'ora non sono presenti nell'engine. Parlo, ad esempio, della possibilità di richiamare MaterialFunction dal codice HLSL, citata nel capitolo \ref{subsec:2_2_CM-MFproblems}, oppure che il materiale per il rendering di una spotlight (ovvero il materiale Light) possa mandare in output un solo colore alla volta, costringendo a creare tre fonti di luce (una per colore) per ogni singolo emettitore della fixture (triplicando così il costo di rendering). \cite{rgbSpotlight}

%TODO Check for more problems!

\subsection{Sviluppi futuri}\label{subsec:6_newDevelops}
Poiché nel corso del tirocinio la maggior parte del tempo è stato speso in un processo di refactoring delle classi principali al progetto (Processo sia lungo per se, che per i problemi riscontrati sopra elencati), l'implementazione di nuove features è stata piuttosto limitata. Questo progetto può avvicinarsi ad uno stato di completamento solamente quando saranno implementate tutte le features disponibili sui fari claypaky. Quelle restanti sono
\begin{itemize}
    \item Prismi
    \item Focus con differenti piani focali
    \item Ruota animazione
    %TODO
\end{itemize}
Tuttavia, con il nuovo sistema di rendering e di gestione dei componenti è molto semplice l'aggiunta di queste nuove features: l'unica difficoltà rimanente è nell'implementazione della feature stessa.

\subsubsection{Arri e Epic Games}\label{subsec:6_1_Arri-EG}
Durante il periodo di tirocinio (Partito a inizio Dicembre 2022 e terminato a fine Luglio 2023) ClayPaky è stata acquisita dal gruppo Arri, dopo essere stata ceduta dal gruppo OSRAM per motivi legati alla politica di vendita di prodotti OEM. Arri è una azienda leader nella vendita di lenti, videocamere e luci per il mondo cinematografico. Durante l'ultimo anno si è incentrata anche nello sviluppo di set cinematografici virtuali creati su UnrealEngine e \say{proiettati} dal vivo su LEDwall \cite{virtualProduction}, piuttosto di utilizzare il tradizionale sistema a green-screen e post produzione. Nello sviluppare questa tecnologia Arri ha collaborato e collabora tutt'oggi con la Epic Games, in modo da poter risolvere rapidamente le problematiche che si presentano nella creazione del loro plugin per la virtualizzazione. Durante il mio periodo da tirocinante, ho personalmente fatto molte riunioni con il dipartimento Virtual della Arri in cui ho illustrato lo stato dell'arte di questo progetto. La Arri ci ha quindi fornito un contatto diretto con la Epic Games a cui abbiamo inviato la nostra repository su github in attesa che possa essere revisionata ed iniziare così un processo di merge del nostro plugin all'interno di UnrealEngine stesso.

\clearpage
\subsection{Ringraziamenti}\label{subsec:6_thank}
Grazie a tutti
\newline
\textcolor{white}{In realtà, siccome sono sentimentale, ho un bel po' di persone da ringraziare. Primi fra tutti i miei amici e colleghi della Homework Heroes Gang, che mi hanno supportato e sopportato per questi 4 anni e dato una grossa mano nello studio degli esami, anche nei periodi più brutti. È stato un onore conoscervi ed avervi come compagni, vi ricorderò per sempre e ricorderò per sempre le speedrun fatte 3 minuti prima della consegna degli homework. Inoltre volevo ringraziare i miei amici Gonfalonieri irl e non, che mi alleviano le giornate ormai da 10 anni a questa parte. Grazie per le prese in giro, per gli scherzi, per le notti su discord e per tanto tanto altro. Poi, ma non in ordine di importanza, volevo ringraziare la mia Lulù che è entrata a piede teso nella mia vita e mi ha fatto capire cosa c'è davvero di bello. Ringrazio anche tutti i miei colleghi in Claypaky (sia del mio ufficio, che non) che hanno contribuito a creare il miglior ambiente lavorativo che abbia mai provato sulla mia pelle da quando ho iniziato a lavorare nel lontano 2017. È stato davvero bello passare ogni giornata con voi, scherzare in ufficio, andare a pranzo, ridere dei colleghi più \_strani\_ e così via. Siete stati tutti dei grandissimi amici ed ispiratori. Passerei ancora ore ad ascoltarvi parlare in ufficio di tutti i vari problemi che si possono generare durante la creazione di un prodotto come quelli di questo campo. Menzione speciale per i miei amici Lorenzo, Tommaso e Davide, che sono state le persone che mi hanno materialmente più supportato durante tutti i miei studi e che mi hanno incoraggiato a intraprendere questo percorso universitario. Ringrazio tante persone quali i miei colleghi tecnici al teatro Domma, gli altri membri dei TRX e mhackeroni, i miei colleghi dell'aps furizon, che mi hanno da sempre stimolato in vari campi ad essere il più creativo possibile e che mi hanno da sempre fatto inseguire i miei sogni. Ringrazio i miei genitori che mi hanno da sempre spinto lungo tutto questo percorso e che si sono subiti per la prima volta la mia lontananza per un periodo di tempo così lungo e filato (preparatevi che quando entreranno i tour grossi, 8 mesi saranno nulla) senza intromettersi troppo nella distanza :D. Ringrazio infine ogni singola persona che ha fatto parte, anche solo per un momento di 10 anni fa, della mia vita. Io credo fermamente che la vita sia un muro fatto di tanti mattoncini deposti da ogni persona che si incontra. C'è chi lascia un mattone più grande e chi uno più piccolo, ma tutti insieme fanno me. E se sono arrivato fino a dove sono ora, lavorativamente e \say{studescamente} parlando, è grazie a davvero tutti. Vi voglio bene.}
\clearpage


\end{document}